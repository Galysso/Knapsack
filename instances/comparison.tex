% Comparison between the new parametric method and (Cerqueus et al., 2015) for the exact computation of the OCSUB
\documentclass[11pt,a4paper]{article}
\usepackage{a4wide}
\usepackage{multirow}
\begin{document}
\begin{table}
\begin{tabular}{c|cc|}
\hline
Instance & New parametric method & (Cerqueus et al., 2015)\\
\hline
A1.DAT & 
0.020
& 
0.216
\\
A2.DAT & 
0.004
& 
0.008
\\
A3.DAT & 
0.008
& 
0.124
\\
A4.DAT & 
0.000
& 
0.012
\\
D1.DAT & 
0.000
& 
0.000
\\
D2.DAT & 
0.000
& 
0.000
\\
D3.DAT & 
0.000
& 
0.000
\\
D4.DAT & 
0.000
& 
0.004
\\
kp28.DAT & 
0.004
& 
0.012
\\
kp28-2.DAT & 
0.000
& 
0.008
\\
kp28c1W-c2ZTL.DAT & 
0.000
& 
0.004
\\
kp28cW-WZTL.DAT & 
0.004
& 
0.012
\\
kp28W.DAT & 
0.000
& 
0.012
\\
kp28W-Perm.DAT & 
0.004
& 
0.008
\\
kp28W-ZTL.DAT & 
0.000
& 
0.000
\\
ZTL28.DAT & 
0.000
& 
0.008
\\
ZTL100.DAT & 
0.004
& 
0.088
\\
ZTL105.DAT & 
0.020
& 
0.296
\\
ZTL250.DAT & 
0.048
& 
0.604
\\
ZTL500.DAT & 
0.144
& 
4.764
\\
ZTL750.DAT & 
0.420
& 
106.86
\\
W7BI-rnd1-1800.DAT & 
0.020
& 
0.236
\\
W7BI-rnd1-3000.DAT & 
0.008
& 
0.084
\\
W7BI-tube1-1800.DAT & 
0.008
& 
0.236
\\
W7BI-tube1-3000.DAT & 
0.004
& 
0.020
\\
W7BI-tube1-asyn.DAT & 
0.000
& 
0.004
\\
W7BI-tube2-1800.DAT & 
0.012
& 
0.208
\\
Wcollage-tube.DAT & 
0.016
& 
0.812
\\
\hline
\end{tabular}
\end{table}
\end{document}
