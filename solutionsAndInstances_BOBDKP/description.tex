\documentclass[11pt]{amsart}
\usepackage{geometry}                % See geometry.pdf to learn the layout options. There are lots.
\geometry{letterpaper}                   % ... or a4paper or a5paper or ... 
%\geometry{landscape}                % Activate for for rotated page geometry
%\usepackage[parfill]{parskip}    % Activate to begin paragraphs with an empty line rather than an indent
\usepackage{graphicx}
\usepackage{amssymb}
\usepackage{epstopdf}
\usepackage{a4wide}
\usepackage{amsmath}
\usepackage{amsthm}
\usepackage{amsfonts}
\usepackage{enumerate}
\let\chapter\section
\usepackage[linesnumbered,algoruled,ruled,vlined]{algorithm2e}
\usepackage{verbatim}
\usepackage{subfig}
\usepackage{tikz}
\usetikzlibrary{calc}
\usetikzlibrary{patterns}
\usepackage{pdflscape}
\usepackage{booktabs}
\usepackage{array}
\usepackage{multirow}
\usepackage{url}
\DeclareGraphicsRule{.tif}{png}{.png}{`convert #1 `dirname #1`/`basename #1 .tif`.png}



\DeclareMathOperator{\BDKP}{2\mathit{DKP}}
\DeclareMathOperator{\BOBDKP}{2\mathit{O}2\mathit{DKP}}

\title{}
\author{}
%\date{}                                           % Activate to display a given date or no date

\begin{document}
\maketitle
%\section{}
%\subsection{}

%
\begin{description}
\item[Group 1:]  6 instances of various sizes, randomly generated with the same generator, without any specific correlation. \texttt{ZTL100, ZTL250, ZTL500 and ZTL750} are 4 instances picked from the collection maintained by Zitzler and Laumanns\footnote{\url{http://www.tik.ee.ethz.ch/sop/download/supplementary/testProblemSuite/}}.
\texttt{ZTL28 and ZTL105} are 2 additional instances derived respectively from ZTL100 and ZTL250, where 
$\omega_i \approx 0.5 \, \sum_{j=1}^n{w_{ij}}, \, i=1,\dots,m$.
\texttt{ZTL150\_f250, ZTL200\_f250} are 2 additional instances derived from ZTL250, where 
$\omega_i \approx 0.5 \, \sum_{j=1}^n{w_{ij}}, \, i=1,\dots,m$.
\texttt{ZTL150\_f500, ZTL200\_f500} are 2 additional instances derived from ZTL500, where 
$\omega_i \approx 0.5 \, \sum_{j=1}^n{w_{ij}}, \, i=1,\dots,m$.
%the capacity is set to half the sum of weight for all the items belonging to the reduced instances. 
The number following ZTL in the name gives the number of variables.
%
\item[Group 2:] 15 correlated instances with 28 variables, introduced by Olga Perederieiva \cite{Perederieiva:2011a}. \texttt{A1, A2, A3, A4, D1, D2, D3, D4, kp28W-ZTL, kp28,  kp28-2, kp28W, kp28W-Perm, kp28c1W-c2ZTL} and \\ \texttt{kp28cW-WZTL} are extention of $\BDKP$ instances available on the OR-library\footnote{\url{http://people.brunel.ac.uk/~mastjjb/jeb/orlib/mknapinfo.html}} where a second objective has been added with respect to several definitions of correlation to obtain a $\BOBDKP$ (see Olga Perederieiva \cite{Perederieiva:2011a} for further details).
%
\item[Group 3:] 7 additional correlated instances with 105 variables, obtained following the rules described for the Group~2. \texttt{W7BI-rnd1-1800, W7BI-rnd1-3000, W7BI-tube1-1800, W7BI-tube1-3000,} \\ \texttt{W7BI-tube1-asyn, W7BI-tube2-1800, Wcollage-tube} are summarized in Table \ref{tab:instances}. 
\end{description}

\begin{table}[ht]
	\centering
	\caption{Information about instances. Column (a) gives the name of the original single-objective $\BDKP$.
Column (b) indicates the manner for generating the second objective: negatively correlated with the first one or, done according to the method reported in Osorio 2005 \cite{Osorio:2005}. %(denoted by \emph{Osorio's method}).
Column (c) reports   the ratio $\omega_1 / \sum_{j=1}^n{w_{1j}}$ for the first dimension (the first constraint). The column (d) is similar to the column (c), for the second dimension.}
	\label{tab:instances}
	\begin{tabular}{llllll}
		\toprule
		instance 			& (a) 		& (b)  & (c)	& (d) 		\\
		\midrule
	%	kp28-2				& kp28		& 28	& Osorio's method			& 0.53				& 0.60					\\
		W7BI-rnd1-1800		& Weing7		& Osorio's method			& 0.48				& 0.50					\\
		W7BI-rnd1-3000		& Weing7		& Osorio's method			& 0.80				& 0.84					\\
		W7BI-tube1-1800		& Weing7	 	& anti-correlated			& 0.50				& 0.50					\\
		W7BI-tube1-3000		& Weing7	 	& anti-correlated			& 0.80				& 0.84					\\
		W7BI-tube1-asyn		& Weing7	 	& anti-correlated			& 0.80				& 0.20					\\
		W7BI-tube2-1800		& Weing7	 	& anti-correlated			& 0.48				& 0.50					\\
		Wcollage-tube		& Weing1	 	& anti-correlated			& 0.50				& 0.50					\\
		\bottomrule
	\end{tabular}
\end{table}

\end{document}  